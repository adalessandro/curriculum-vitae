\documentclass{simplecv}
\usepackage[latin1]{inputenc}
\usepackage[spanish]{babel}
\usepackage{hyperref}
\usepackage{url}

\usepackage{multibib}
\newcites{conferences}{{Papers in refereed conferences}}
\newcites{journals}{{Journals}}

\begin{document}

\leftheader{
\textbf{ariel@vanguardiasur.com.ar}\\
\small{\url{http://github.com/adalessandro}}\\
\small{\url{http://linkedin.com/in/ariel-dalessandro}}\\
}

\rightheader{
1160 Mitre Street \#3A\\
S2000 Rosario (Argentina)\\
+54~9~(0341)~2616~873
}

\title{Ariel D'Alessandro}

\maketitle

Working as Software Developer since 2011, involved in many different projects
professionally, dedicated to Open Source since 2014. Got a Computer Science
degree in 2017, with a degree thesis on Stereo-visual SLAM systems.

\section{Professional Experience}

\begin{topic}

% Job entry
\item [Aug 2020 - now] \emph{Software Engineer} -
iRobot, LLC (\url{http://www.irobot.com/})

Robotics company developing autonomous cleaning devices. Currently working on
autologging reliability and testing automation.
\begin{itemize}
\item \textbf{Technologies}: C++, Python.
\end{itemize}

% Job entry
\item [Apr 2020 - Jun 2020] \emph{Freelance} -
Leistung, SRL (\url{https://leistungargentina.com.ar/})

Worked as a freelancer with Leistung SRL, a local company focused on development
and production of medical devices for mechanical ventilation.
As a freelancer, was in permanent contact with the client, determining
requirements and tasks.
\begin{itemize}
\item Solved a persistent issue on client's Qt5 app, which was caused by a
deadlock bug in libX11 userspace library. See
\url{https://gitlab.freedesktop.org/xorg/lib/libx11/-/merge\_requests/29}
\item Yocto BSP customization from scratch. Added specific layer addressing
client requirements on top of Digi Embedded Yocto.
\item \textbf{Technologies}: libX11, Yocto, Python, Bash.
\end{itemize}

% Job entry
\item [Sep 2016 - Sep 2019] \emph{Software Engineer} -
MotionFigures, LLC (\url{http://www.motionfigures.com/})

Robotics startup company. During this project, worked as one of the first
employees, in charge of the entire software stack. Tasks included on-site work
to bring-up the initial prototype and debug hardware issues.

\begin{itemize}

\item BSP using Buildroot, U-boot and Linux Kernel for a custom Sunxi AllWinner
A20 board.
\begin{itemize}
\item Developed custom UART driver and 1-wire half-duplex protocol design to
discover, communicate and drive servo endpoints.
\item IMU BNO055 IIO driver rework for quaternion estimation using sensor fusion
in user-space app.
\item Worked on audio codec drivers to support Sunxi A20 SoC and meet product
requirements.
\item \textbf{Technologies}: ARM, Sunxi Allwinner A20, UART, 1-wire, IMU BNO055
IIO, sound subsystem, Linux Kernel, U-boot, Buildroot, C, Bash.
\end{itemize}

\item Worked on core user-space app using Node JS.
\begin{itemize}
\item Developed Node JS addons in C++ for low-level hardware communication with
custom UART buses and IMU.
\item Research and implementation of 3D geometry models for robotic motion,
including inverse kinematic solvers and IMU sensor fusion algorithms.
\item Developed JS based apps using graphics library three.js and d3. Added
physics simulation using DART and pybullet engines.
\item \textbf{Technologies}: Robotic motion, inverse kinematics, IMU sensor
fusion, physics engines, DART, pybullet, C++, Node JS, three.js, d3.js.
\end{itemize}

\item Programmed several MCUs using the following libraries/frameworks: STM8
cosmic, STM32 libopencm3/unicore-mx, ESP32 esp-idf.
\begin{itemize}
\item Implemented custom UART and 1-wire half-duplex protocol.
\item PID controllers for servo motor driving.
\item Fixed-point arithmetic support for non-FPU MCUs.
\item Ported the entire 3D geometry models for robotic motion to STM32 board.
\item \textbf{Technologies}: C, STMicroelectronics STM8 cosmic, STM32
libopencm3, STM32 unicore-mx; Espressif ESP32 esp-idf; UART, 1-wire, PID
controller, embedded robotic motion.
\end{itemize}

\item Research on machine learning techniques applied to robotic motion.
\begin{itemize}
\item Implemented reinforcement learning model using OpenAI gym/roboschool
projects.
\item \textbf{Technologies}: Machine learning, reinforcement learning, OpenAI,
Python.
\end{itemize}

\end{itemize}

% Job entry
\item [July 2015 - June 2017 and April 2020 - June 2020]
\emph{Software Engineer} -
VanguardiaSur (\url{http://www.vanguardiasur.com.ar/})

Company providing Embedded Linux and Linux Kernel consulting services to
customers all over the world.

\begin{itemize}

\item Project June 2015 - Feb 2016
\begin{itemize}
\item Developed kernel drivers for NXP LPC18xx/43xx, accepted upstream.
\item \textbf{Technologies}: ARM, NXP LPC18xx/43xx, Linux Kernel, C, Git.
\end{itemize}

\item Project: July 2015 - Feb 2016
\begin{itemize}
\item Worked on two different products from client, based on Sunxi Allwinner A20
and Rockchip RK3288 SoCs. Tasks included adding BSP support and customization
using Buildroot and OpenWRT, along with U-boot and Linux kernel.
\item \textbf{Technologies}: ARM, Sunxi Allwinner A20, Rockchip RK3288, Linux
Kernel, U-boot, Buildroot, OpenWRT, C, Bash.
\end{itemize}

\item Project: Aug 2015 - Oct 2015
\begin{itemize}
\item Worked on a startup project using a Renesas RZ/A1L custom board.
\item Prepared BSP using Buildroot, U-boot and Linux kernel to meet client
requirements, with special emphasis on low memory footprint. Developed IMU
MPU9250 user-space library.
\item \textbf{Technologies}: ARM, Renesas RZ/A1L, IMU MPU9250, Linux Kernel,
U-boot, Buildroot, C, Bash.
\end{itemize}

\item Project: Nov 2015 - June 2017
\begin{itemize}
\item Mantained a BSP (Buildroot + U-boot + Linux kernel) for a product based on
Texas Instruments AM335x. Worked on the user-space side developing several
tools.
\item \textbf{Technologies}: ARM, Texas Instruments AM335x, Linux Kernel,
U-boot, Buildroot, C, C++, Bash, Qt.
\end{itemize}

\end{itemize}

% Job entry
\item [Oct 2012 - June 2015] \emph{Software Developer} -
Ministry of Health of Santa Fe (\url{http://www.santafe.gob.ar/})

Developed and maintained systems for health public administration using
Microsoft Visual Basic, Python/Django and PHP/Symfony2.

\textbf{Technologies}: Bash, Visual Basic, Python, PHP, JS, SQL, SVN.

% Job entry
\item [April 2011 - Oct 2011] \emph{Software Developer} -
Elepe Servicios (\url{http://www.elepeservicios.com.ar/})

Developed and maintained systems for business administration using Genexus Ev.1
(Web) and AllFusion Plex.

\textbf{Technologies}: Genexus, AllFusion Plex, SQL, SVN.

\end{topic}

\section{Open source contributions}

\begin{topic}

\item [Linux Kernel]
\url{https://git.kernel.org/pub/scm/linux/kernel/git/torvalds/linux.git/log/?qt=author&q=Ariel+D.*Alessandro}

\item [Buildroot]
\url{https://git.busybox.net/buildroot/log/?qt=author&q=Ariel+D.*Alessandro}

\item [U-boot]
\url{https://github.com/u-boot/u-boot/commits?author=adalessandro}

\item [UniCore Mx]
\url{https://github.com/insane-adding-machines/unicore-mx-examples/commits?author=adalessandro}

\item [v4l-utils (in progress)]
\url{https://www.spinics.net/lists/linux-media/msg174577.html}

\end{topic}

\section{Computer Science Degree}

\begin{topic}

\item [2008 - 2017] \emph{Computer Science Degree} - National University of
Rosario (\url{http://www.unr.edu.ar/}) - GPA: 9.00

\item [Dec 2017] \emph{Computer Science Degree Thesis} - CIFASIS
(\url{http://www.cifasis-conicet.gov.ar/})

\textbf{Title}: \emph{Real-time dense mapping for Stereo Visual SLAM systems}

\textbf{Supervisors}: \emph{Taih� Pire} and \emph{Rodrigo Baravalle}

\textbf{Abstract}:
A robot should be able to estimate an accurate and dense 3D model of its
environment (a map), along with its pose relative to it, in order to be able to
navigate autonomously without collisions.
As the robot moves from its starting position and the estimated map grows, the
computational and memory footprint of a dense 3D map increases and might exceed
the robot capabilities in a short time.
However, a global map is still needed to maintain its consistency and plan for
distant goals, possibly out of the robot field of view.
In this work, we address such problem by proposing a
\emph{real-time stereo mapping} pipeline, feasible for standard CPUs, which is
locally dense and globally sparse and accurate.
Our algorithm is based on a graph relating poses and salient visual points, in
order to maintain a long-term accuracy with a small cost.
Within such framework, we propose an efficient dense fusion of several stereo
depths in the locality of the current robot pose.
Code released as a ROS package under GPLv3 license:
\href{http://www.github.com/CIFASIS/dense-sptam}{Dense S-PTAM}.

\end{topic}

\section{Publications}

\bibliographystyleconferences{plain}
\nociteconferences{hacia-una-densificacion}
\bibliographyconferences{conferences}

\bibliographystylejournals{plain}
\nocitejournals{real-time-dense}
\bibliographyjournals{journals}

\section{Teaching/Courses}

\begin{topic}

\item [Apr 2016 - July 2016] \emph{Teacher Assistant} -
National University of Rosario - Engineering, Programming

\textbf{Topics}: C, Fortran, Pascal. Basic Data Structures. Operating system
concepts.

\item [Aug 2015] \emph{Workshop: Booting Linux on a CIAA NXP}

SASE 2015 (Simposio Argentino de Sistemas embebidos)
\url{http://www.sase.com.ar/} - National University of Buenos Aires

Introducing CIAA NXP board bring-up (and beyond) using Linux, U-boot and
Buildroot.

\end{topic}

\section{Languages}

\begin{itemize}

\item \textbf{Spanish}: Native - Mother tongue.

\item \textbf{English}: Advanced - Fluent, written and spoken.

\end{itemize}

\end{document}
